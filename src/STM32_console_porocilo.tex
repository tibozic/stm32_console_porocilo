\documentclass{article}

%
% packages
%
% references
\usepackage[backend=biber]{biblatex}
% better padding
\usepackage[margin=100px]{geometry}
% images
\usepackage{graphicx}
% math symbols
\usepackage{amsmath}
% more symbols
\usepackage{textcomp}
% more math symbols
\usepackage{amssymb}
% links
\usepackage{hyperref}
% linsk jump to top of figures
\usepackage[all]{hypcap}
% floating options ([H])
\usepackage{float}
% source code
\usepackage{listings}
% indent first paragraph after section
\usepackage{indentfirst}
% continous enumeration
\usepackage{enumitem}
% cmark and xmark
\usepackage{pifont}
% colors
\usepackage[table,x11names]{xcolor}
% column style
\usepackage{tabu}
% section settings
\usepackage{titlesec}
% colors
\usepackage{xcolor}

\addbibresource{~/scripts/latex/ref.bib}

%
% custom commands
%
\newcommand{\N}{\mathbb{N}}
\newcommand{\lxor}{\veebar}
\newcommand{\cmark}{\ding{51}}
\newcommand{\xmark}{\ding{55}}

% path for images
\graphicspath{{images/}}

% link stuff
\hypersetup{
	colorlinks=true,
	linkcolor=blue,
	filecolor=red,
	runcolor=red,
	urlcolor=cyan,
	anchorcolor=black,
	citecolor=.,
	menucolor=red,
	pdftitle={STM32 Igralna konzola},
}

\definecolor{codegreen}{rgb}{0,0.6,0}
\definecolor{codegray}{rgb}{0.5,0.5,0.5}
\definecolor{codepurple}{rgb}{0.58,0,0.82}
\definecolor{backcolour}{rgb}{0.95,0.95,0.92}
% code stuff
\lstdefinestyle{mystyle}{
    backgroundcolor=\color{backcolour},   
    commentstyle=\color{codegreen},
    keywordstyle=\color{magenta},
    numberstyle=\tiny\color{codegray},
    stringstyle=\color{codepurple},
    basicstyle=\ttfamily\footnotesize,
    breakatwhitespace=false,         
    breaklines=true,                 
    captionpos=b,                    
    keepspaces=true,                 
    numbers=left,                    
    numbersep=5pt,                  
    showspaces=false,                
    showstringspaces=false,
    showtabs=false,                  
    tabsize=2
}
\lstset{style=mystyle}

% Extra subsections (parargaph)
\setcounter{secnumdepth}{4}

\titleformat{\paragraph}
{\normalfont\normalsize\bfseries}{\theparagraph}{1em}{}
\titlespacing*{\paragraph}
{0pt}{3.25ex plus 1ex minus .2ex}{1.5ex plus .2ex}

% programming stuff
\lstset{language=Python}


\author{Timotej Bo\v{z}i\v{c}, Nik Juki\v{c}}
\title{STM32 Igralna Konzola}
\date{2022}

\begin{document}

\maketitle

\tableofcontents

\clearpage

\section{Uvod}
\noindent
Vsa koda projekta je dostopna na github: \href{https://github.com/tibozic/stm32_console}{https://github.com/tibozic/stm32\_console} \\
Vsa koda poro\v{c}ila z videi je dostopna na github: \href{https://github.com/tibozic/stm32_console_porocilo}{https://github.com/tibozic/stm32\_console\_porocilo} \\

\noindent
Parts list:
\begin{itemize}
  \item STM32H750 Discovery kit
\end{itemize}

\noindent
Za\v{c}etni cilj je bil vzpostavitev vgrajenega zaslona na
plo\v{s}\v{c}i brez kakr\v{s}nekoli knji\v{z}nice in potem napisati
neko osnovno knji\v{z}nico za uporabo grafi\v{c}nih elementov
(risanje \v{c}rt, kvadratov, trikotnikov). \\
Ker nama po kak\v{s}nem tednu raziskovanja, branja dokumentacije in testiranja
ni uspelo narediti konkretnega napredka, sva se
odlo\v{c}ila, da raje uporabiva knji\v{z}nico za vzpostavitev
zaslona in risanje elementov, midva pa se bova bolj osredoto\v{c}ila
na kon\v{c}en produkt, ki naj bi bil osnovna igralna konzola z nekaj
enostavnimi igrami. \\
Ker ima plo\v{s}\v{c}a, ki jo uporabljava na voljo zaslon z mo\v{z}nostjo
detekcije dotika, sva so odlo\v{c}ila da uporabiva to namesto ``joystick'',
saj v tem primeru, za implementacijo nebi potrebovala nobenih dodatnih
delov. \\

\noindent
Knji\v{z}nico, ki sva jo uporabila je TouchGFX, ki je opitimiziran za
uporabo z napravami STM32. Knji\v{z}nica je prilo\v{z}ena tudi z
TouchGFX Designer, ki je orodje za grafi\v{c}no oblikovanje in deluje
na principu ``What You See Is What You Get'' in omogo\v{c}a tudi
``Drag \& Drop'' elementov. \\

\section{Lastnosti TouchGFX}
\subsection{TouchGFX Designer}
\noindent
TouchGFX Designer je zelo uporaben za osnovne in stati\v{c}ne
strani, za kaj kompleksnej\v{s}ega pa je potrebna
dodatna koda. \\

\begin{figure}[H] % H means exactly in place
  \centering
  \includegraphics[width=0.6\textwidth]{elements.PNG}
  \caption{Menu za dodajanje elementov}
\end{figure}

\begin{figure}[H] % H means exactly in place
  \centering
  \includegraphics[width=0.2\textwidth]{buttons.PNG}
  \includegraphics[width=0.2\textwidth]{images.PNG}
  \includegraphics[width=0.2\textwidth]{shapes.PNG}
  \caption{Gumbi, slike in oblike}
\end{figure}

\begin{figure}[H] % H means exactly in place
  \centering
  \includegraphics[width=0.8\textwidth]{screen.PNG}
  \caption{TouchGFX Designer celoten zaslon}
\end{figure}

\noindent
Ko zaklju\v{c}imo z oblikovanjem zaslona, pritisnemo gumb ``simulate''
v spodnjem desnem delu zaslona. To lahko naredimo le \v{c}e nismo
uporabili nobenih zunanjih spremenljivk (zunanjih gumbov, LED, itd.).
Ko to naredimo se bo na\v{s} projekt zgradil in ga lahko sedaj
preizkusimo. \\
V primeru uporabe zunanjih spremenljivk, pa
moramo na\v{s} projekt zgraditi in nalo\v{z}iti na plo\v{s}\v{c}ico,
ker nam v tem primeru moznost simulacije ni na voljo. To storimo tako,
da pritisnemo gumb generate v spodnjem desnem kotu.
Program nato generira kodo za na\v{s} dizajn. Projekt lahko sedaj
odpremo v CubeIDE, kjer lahko spreminjamo kodo ali pa nalo\v{z}imo
program na plo\v{s}\v{c}ico. \\

\noindent
Ko je projekt generiran, se ustvarijo dve glavni mapi:
\begin{itemize}
    \item generated
    \item gui
\end{itemize}

\begin{figure}[H] % H means exactly in place
  \centering
  \includegraphics[width=0.25\textwidth]{closed.PNG}
  \caption{Generirane datoteke}
\end{figure}

\noindent
V mapi generated so definirani vsi elementi, ki smo jih ustvarili
v TouchGFX Designer, slike ki smo jih dodali ter ``Base'' datoteke
za vse poglede, ki smo jih ustvarili. V tej mapi ne smemo spreminjati
nobenih datotek, saj bodo na\v{s}e spremembe povo\v{z}ene vsaki\v{c}
ko spremenimo kaj v Designer-ju. \\
\begin{figure}[H] % H means exactly in place
  \centering
  \includegraphics[width=0.4\textwidth]{generated.PNG}
  \caption{Mapa generated}
\end{figure}

\noindent
V mapi gui pa so datoteke za vsak pogled, ki smo ga ustvarili. Te
datoteke so namenjene na\v{s}emu urejanju.
\begin{figure}[H] % H means exactly in place
  \centering
  \includegraphics[width=0.4\textwidth]{gui.PNG}
  \caption{Mapa gui}
\end{figure}

\subsection{TouchGFX Library}
\noindent
Ko je projekt zgeneriran je uporaba knji\v{z}nice dokaj
preprosta. Na objekte kli\v{c}emo funkcije kot so .setXY(),
.setVisible(), ali pa inicializiramo nove objekte s funkcijami kot
so Box(). \\
Ko nek objekt posodobimo ga mora\v{s} ozna\v{c}iti za ``redraw'' s
funkcijo .invalidate(). \\
Glavna slabost te knji\v{z}nice je, da morajo biti vsi elementi na
ekranu stati\v{c}no definirani, to nama je povzro\v{c}alo precej te\v{z}av
pri Snake game-u, saj sva morala maksimalno dol\v{z}ino ka\v{c}e
dolo\v{c}iti stati\v{c}no, in za vsak njen del na za\v{c}etku programa
stati\v{c}no rezervirati prostor. V primeru, da bi
\v{z}elela imeti ka\v{c}o \v{c}ez cel ekran bi potrebovala okoli 250
tiso\v{c} delov rezervirati stati\v{c}no na za\v{c}etku programa. To je
zelo problemati\v{c}no, saj ponavadi vgrajene naprave nimajo na razpolago
veliko pomnilnega prostora, ta pa bi bil zaseden skozi celoten \v{z}ivljenski
cikel programa. Ta problem sva re\v{s}ila tako, da sva pove\v{c}ala \v{s}irino
ka\v{c}e, kar je zmanj\v{s}alo \v{s}tevilo potrebnih delov na
manj kot 5000, kar \v{s}e vedno ni idealno, je pa izvedljivo.


\section{Main Menu}
\noindent
Izoblikovala sva preprost meni, preko katerega lahko igralec z dotikom
izbira med vsemi igrami, ki so na voljo. \\
\begin{figure}[H] % H means exactly in place
  \centering
  \includegraphics[width=0.4\textwidth]{menu.jpg}
  \caption{Main menu}
\end{figure}

\section{Igre}
\noindent
Pri izboru iger, sva se omejila na tiste, ki bi bile relativno preproste
za implementacijo, in med katerimi je prisotna raznolikost v samem
igranju. Na koncu sva si izbrala tri ``retro'' igre in sku\v{s}ala
ustvariti \v{c}im bolj zvesto rekreacijo.
\begin{itemize}
  \item Snake Game
  \item Tic Tac Toe
\end{itemize}

\subsection{Snake Game}
\noindent
V tej igri nadzorujemo premikanje ka\v{c}e po 480x272 plo\v{s}\v{c}ic
velikem prostoru. Po prostoru se naklju\v{c}no generira ko\v{s}\v{c}ek
hrane. Ka\v{c}o premikamo s pritiski na smerne gumbe. \\

\noindent
Cilj igre je s ka\v{c}o pojesti \v{c}im ve\v{c} hrane brez da se ka\v{c}a
zaleti sama vase ali v robove ekrana. \\
Video: \href{https://github.com/tibozic/stm32_console_porocilo/tree/master}{https://github.com/tibozic/stm32\_console\_porocilo/tree/master}

\subsubsection{Koda igre}
\noindent
Inicializacija spremenljivk.
\begin{lstlisting}[language=c++]
bool game_started = false;
short snake_direction = SNAKE_DOWN;
int snake_length = 1;

snake_piece *head = NULL;
snake_piece *tail = NULL;

touchgfx::Box snake_pixels[MAX_SNAKE_PIECES];
\end{lstlisting}

\noindent
Inicializacija za\v{c}etnega stanja igre, kar vklju\v{c}uje:
\begin{itemize}
    \item za\v{c}etna pozicija in orientacija ka\v{c}e
    \item skritje gumbov in oken prej\v{s}nih instanc
\end{itemize}
\begin{lstlisting}[language=c++]
void screen_snake_gameView::game_snake_start()
{
	// first hide the button
	btn_snake_start.setVisible(false);
	btn_snake_start.invalidate();

	btn_back.setVisible(false);
	btn_back.invalidate();

	snake_head.setVisible(false);
	snake_head.invalidate();

	lbl_game_over.setVisible(false);
	lbl_game_over.invalidate();

	snake_pixels[0] = Box();

	if( head == NULL )
		head = (snake_piece*)malloc(sizeof(snake_piece));

	if( head == NULL ) {
		error();
		return;
	}

	head->pixel = &snake_pixels[0];

	head->pixel->setPosition(10, 20, 10, 10);
	head->pixel->setColor(touchgfx::Color::getColorFromRGB(0, 255, 50));
	add(*head->pixel);
	head->pixel->getParent()->invalidate();

	head->next = NULL;
	head->prev = NULL;
	head->old_x = 0;
	head->old_y = 0;

	tail = head;

	snake_direction = SNAKE_DOWN;

	snake_length = 1;

	game_started = true;

	// game_snake_loop();
}
\end{lstlisting}

\noindent
Vecina kode se nahaja v glavni funkciji, ki je v bistvu glavna
zanka celotne igre. \\

\noindent
Pregledamo \v{c}e je glava ka\v{c}e izven igralnega polja
\begin{lstlisting}[language=c++]
if( head->pixel->getX() > SCREEN_WIDTH || head->pixel->getX() < 0 ||
    head->pixel->getY() > SCREEN_HEIGHT || head->pixel->getY() < 0)
{
    game_started = false;
}
\end{lstlisting}

\noindent
Pregledamo \v{c}e se je ka\v{c}a zaletela sama vase
\begin{lstlisting}[language=c++]
snake_piece *snake_part = head->next;
while( snake_part != NULL )
{
    if( snake_part->pixel->getX() == head->pixel->getX() && snake_part->pixel->getY() == head->pixel->getY() ) {
        game_started = false;
        break;
    }

    snake_part = snake_part->next;
}
\end{lstlisting}

\noindent
Pregledamo, \v{c}e je ka\v{c}a pojedla hrano
\begin{lstlisting}[language=c++]
snake_part = head;

while( snake_part != NULL )
{
    if( snake_part->pixel->getX() == food.getX() && snake_part->pixel->getY() == food.getY() ) {
      // ...
    }
    snake_part = snake_part->next;
}
\end{lstlisting}

\noindent
V primeru, da je ka\v{c}a pojedla hrano, jo moramo podalj\v{s}ati (dodati
nov del)
\begin{lstlisting}[language=c++]
snake_piece *new_piece = (snake_piece*)malloc(sizeof(snake_piece));

if( new_piece == NULL ) {
    error();
    return;
}

snake_pixels[snake_length-1] = Box();
new_piece->pixel = &snake_pixels[snake_length-1];
new_piece->pixel->setPosition(0, 0, PIXEL_WIDTH, PIXEL_HEIGHT);
new_piece->pixel->setColor(touchgfx::Color::getColorFromRGB(255, 130, 0));
new_piece->pixel->setVisible(true);
add(*new_piece->pixel);
new_piece->pixel->getParent()->invalidate();

tail->next = new_piece;
new_piece->prev = tail;
new_piece->next = NULL;
new_piece->old_x = 0;
new_piece->old_y = 0;
tail = new_piece;
\end{lstlisting}

\noindent
\v{C}e je ka\v{c}a pojedla hrano moramo tudi generirati nov kos hrane.
\begin{lstlisting}[language=c++]
int food_new_x = pseudo_random(tick) % SCREEN_WIDTH;
int food_new_y = pseudo_random2(tick) % SCREEN_HEIGHT;

food_new_x = food_new_x - (food_new_x % 10);
food_new_y = food_new_y - (food_new_y % 10);

food.setXY(food_new_x, food_new_y);
food.invalidate();

Unicode::snprintf(lbl_score_valBuffer, LBL_SCORE_VAL_SIZE, "%d", snake_length);
lbl_score_val.invalidate();
\end{lstlisting}

\noindent
Premik ka\v{c}e za eno polje
\begin{lstlisting}[language=c++]
if( snake_direction == SNAKE_RIGHT )
{
    head->old_x = head->pixel->getX();
    head->old_y = head->pixel->getY();
    head->pixel->setXY(head->old_x+(SNAKE_MOVE), head->old_y);
    head->pixel->getParent()->invalidate();

    snake_piece *piece = head->next;
    while( piece != NULL )
    {
        piece->old_x = piece->pixel->getX();
        piece->old_y = piece->pixel->getY();
        piece->pixel->setXY(piece->prev->old_x, piece->prev->old_y);
        piece->pixel->getParent()->invalidate();
        piece = piece->next;
    }
}
else if( snake_direction == SNAKE_LEFT )
{
    head->old_x = head->pixel->getX();
    head->old_y = head->pixel->getY();
    head->pixel->setXY(head->old_x-(SNAKE_MOVE), head->old_y);
    head->pixel->getParent()->invalidate();

    snake_piece *piece = head->next;
    while( piece != NULL )
    {
        piece->old_x = piece->pixel->getX();
        piece->old_y = piece->pixel->getY();
        piece->pixel->setXY(piece->prev->old_x, piece->prev->old_y);
        piece->pixel->getParent()->invalidate();
        piece = piece->next;
    }
}
else if( snake_direction == SNAKE_UP )
{
    head->old_x = head->pixel->getX();
    head->old_y = head->pixel->getY();
    head->pixel->setXY(head->old_x, head->old_y-(SNAKE_MOVE));
    head->pixel->getParent()->invalidate();

    snake_piece *piece = head->next;
    while( piece != NULL )
    {
        piece->old_x = piece->pixel->getX();
        piece->old_y = piece->pixel->getY();
        piece->pixel->setXY(piece->prev->old_x, piece->prev->old_y);
        piece->pixel->getParent()->invalidate();
        piece = piece->next;
    }
}
else if( snake_direction == SNAKE_DOWN )
{
    head->old_x = head->pixel->getX();
    head->old_y = head->pixel->getY();
    head->pixel->setXY(head->old_x, head->old_y+(SNAKE_MOVE));
    head->pixel->getParent()->invalidate();

    snake_piece *piece = head->next;
    while( piece != NULL )
    {
        piece->old_x = piece->pixel->getX();
        piece->old_y = piece->pixel->getY();
        piece->pixel->setXY(piece->prev->old_x, piece->prev->old_y);
        // piece->pixel->setXY(head->old_x, head->old_y+(SNAKE_MOVE));
        piece->pixel->getParent()->invalidate();
        piece = piece->next;
    }
}

head->pixel->getParent()->invalidate();
\end{lstlisting}

\noindent
Ko igralec pritisne gumb za spremembo smeri ka\v{c}e, se kli\v{c}e
naslednja funkcija (za vsako smer druga), ki shrani smer v globalno
spremenljivko.
\begin{lstlisting}[language=c++]
void screen_snake_gameView::change_direction_up() {
	snake_direction = SNAKE_UP;
}
\end{lstlisting}

\noindent
Funkciji za generiranje pozicije novega kosa hrane.
\begin{lstlisting}[language=c++]
int screen_snake_gameView::pseudo_random(int tick) {
	return (3*snake_length + SNAKE_SPEED*3*tick + 3*tail->pixel->getX() * 3*tail->pixel->getY());
}

int screen_snake_gameView::pseudo_random2(int tick) {
	return (7*snake_length + SNAKE_SPEED*7*tick + 7*tail->pixel->getX() * 7*tail->pixel->getY());
}
\end{lstlisting}

\subsection{Tic Tac Toe}
\noindent
To je vsem znana igra za 2 igralca, kjer je cilj biti prvi igralec, ki
spravi 3 enake simbole v vrsto. \\
Video: \href{https://github.com/tibozic/stm32_console_porocilo/tree/master}{https://github.com/tibozic/stm32\_console\_porocilo/tree/master}

\subsubsection{Koda igre}
\noindent
Inicializacija spremenljivk.
\begin{lstlisting}[language=c++]
char board[3][3];
bool turn = true; // true -> player 1, false -> player2
short turn_number = 0;
bool game_over = false;
short result = 0;
\end{lstlisting}

\noindent
Ponastavimo stanje igralne plo\v{s}\v{c}e, v primeru da to ni prva
instanca igre, ki jo igramo. Skrijemo se morebitna okna in gumbe
prej\v{s}nih instanc.
\begin{lstlisting}[language=c++]
screen_tictactoe_gameView::screen_tictactoe_gameView()
{
    for(short i = 0; i < 3; ++i) {
        for(short j = 0; j < 3; ++j) {
            board[i][j] = 0;
        }
    }


    game_over = false;
    lbl_game_over.setVisible(false);
    lbl_game_over.invalidate();

    box_background.setVisible(false);
    box_background.invalidate();

    lbl_result.setVisible(false);
    lbl_result.invalidate();

    btn_back.setVisible(false);
    btn_back.invalidate();

    turn_number = 0;

    turn = true;
}
\end{lstlisting}

\noindent
Za vsako polje moramo voditi posebej spremenljivke in metodo, ki
hranijo podatke o stanju igre in preverjajo \v{c}e se je igra \v{z}e
zaklju\v{c}ila.

\begin{lstlisting}[language=c++]
void screen_tictactoe_gameView::pos1_clicked() {
	if( !game_over && board[0][0] == 0 ) {
		if( turn ) {
			board[0][0] = 'X';
			pos1_cross.setVisible(true);
			pos1_cross.invalidate();
		} else {
			board[0][0] = 'O';
			pos1_circle.setVisible(true);
			pos1_circle.invalidate();
		}

		turn_number++;

		short temp_result = is_game_over(0, 0);

		if( temp_result != 0 ) {
			game_over = true;
			result = temp_result;
			return;
		}

		turn = !turn;
	}
}
\end{lstlisting}

\noindent
V tem delu kode pa glede na stanje spremenljivke game\_over izri\v{s}emo
kon\v{c}ni ekran.
\begin{lstlisting}[language=c++]
void screen_tictactoe_gameView::handleTickEvent() {
	if( game_over ) {
		btn_back.setVisible(true);
		btn_back.invalidate();

		box_background.setVisible(true);
		box_background.invalidate();

		lbl_game_over.setVisible(true);
		lbl_game_over.invalidate();

		lbl_result.setVisible(true);

		if( result == 1 )
			Unicode::snprintf(lbl_resultBuffer, LBL_RESULT_SIZE, "Player 1 wins");
		else if( result == 2 )
			Unicode::snprintf(lbl_resultBuffer, LBL_RESULT_SIZE, "Player 2 wins");
		else if( result == 3 )
			Unicode::snprintf(lbl_resultBuffer, LBL_RESULT_SIZE, "Tie");

		lbl_result.invalidate();
	}
	else {
		if( turn_number == 0 ) {
			btn_back.setVisible(true);
			btn_back.invalidate();
		}
		else {
			btn_back.setVisible(false);
			btn_back.invalidate();
		}

		if( turn )
			Unicode::snprintf(lbl_turnBuffer, LBL_TURN_SIZE, "1");
		else
			Unicode::snprintf(lbl_turnBuffer, LBL_TURN_SIZE, "2");

		lbl_turn.invalidate();
	}
}
\end{lstlisting}

\noindent
Sama metoda, ki se izvede ob pritisku na polje, da se preveri morebiten
konec igre.

\begin{lstlisting}[language=c++]
short screen_tictactoe_gameView::is_game_over(int x, int y) {
	// 0 -> game isn't over
	// 1 -> player 1 wins
	// 2 -> player 2 wins
	// 3 -> tie

	char symbol;

	if(turn)
		symbol = 'X';
	else
		symbol = 'O';

    //check col
    for(short i = 0; i < 3; i++){
        if(board[x][i] != symbol)
            break;
        if(i == 3-1){
            if( turn )
            	return 1;
            else
            	return 2;
        }
    }

    //check row
    for(int i = 0; i < 3; i++){
        if(board[i][y] != symbol)
            break;
        if(i == 3-1){
            if( turn )
            	return 1;
            else
            	return 2;
        }
    }

    //check diagonal
    if(x == y){
        //we're on a diagonal
        for(int i = 0; i < 3; i++){
            if(board[i][i] != symbol)
                break;
            if(i == 3-1){
                if( turn )
                	return 1;
                else
                	return 2;
            }
        }
    }

    //check anti diag
    if(x + y == 3-1){
        for(int i = 0; i < 3; i++){
            if(board[i][(3-1)-i] != symbol)
                break;
            if(i == 3-1){
                if( turn )
                	return 1;
                else
                	return 2;
            }
        }
    }

	if( turn_number == 9 )
		return 3;

	return 0;
}
\end{lstlisting}

\subsection{Space invaders}
\noindent
Cilj igre je odstraniti vse nasprotnike preden pridejo do
igral\v{c}eve ladje. \\
Video:\href{https://github.com/tibozic/stm32_console_porocilo/tree/master}{https://github.com/tibozic/stm32\_console\_porocilo/tree/master}

\subsubsection{Koda igre}
\noindent
Inicializacija spremenljivk
\begin{lstlisting}[language=c++]
touchgfx::ScalableImage enemies[NUM_OF_ENEMIES];
int enemy_move_direction = MOVE_RIGHT;

bool invaders_game_over = false;
int enemies_killed = 0;
\end{lstlisting}

\noindent
Inicializacija zacetnega stanja, ki vsebuje:
\begin{itemize}
  \item Za\v{c}etna pozicija ladje
  \item Skritje gumbov prej\v{s}nih instanc iger
  \item Postavitev vseh nasprotnikov
\end{itemize}
\begin{lstlisting}[language=c++]
screen_space_invadersView::screen_space_invadersView()
{
	// Hide the reference enemy
	enemy1.setVisible(false);
	enemy1.invalidate();

	// hide the bullet
	bullet.setVisible(false);
	bullet.setXY(-1, -1);
	bullet.invalidate();

	bullet_enemy.setVisible(false);
	bullet_enemy.setXY(-1, -1);
	bullet_enemy.invalidate();

	lbl_game_over.setVisible(false);
	lbl_game_over.invalidate();

	btn_back.setVisible(false);
	btn_back.invalidate();

	lbl_lose.setVisible(false);
	lbl_lose.invalidate();

	lbl_win.setVisible(false);
	lbl_win.invalidate();

	enemies_killed = 0;

	enemy_move_direction = MOVE_RIGHT;

	// Initialize all the enemies
	int current_x = 0;
	int current_y = 0;
	int counter = 0;

	for(int i = 0; i < NUM_OF_ROWS_OF_ENEMIES; ++i) {
		current_x = 0;
		for(int j = 0; j < NUM_OF_ENEMIES_PER_ROW; ++j) {
			enemies[counter].setBitmap(touchgfx::Bitmap(BITMAP_SPACE_INVADERS_ENEMY_ID));
			enemies[counter].setPosition(current_x, current_y, 50, 52);
			enemies[counter].setScalingAlgorithm(touchgfx::ScalableImage::NEAREST_NEIGHBOR);
			enemies[counter].setVisible(true);
			add(enemies[counter]);
			enemies[counter].invalidate();

			current_x += enemy1.getWidth() + SPACE_BETWEEN_ENEMIES_X;
			counter++;
		}
		current_y += enemy1.getHeight() + SPACE_BETWEEN_ENEMIES_Y;
	}

	invaders_game_over = false;
}
\end{lstlisting}

\noindent
Vsako urino periodo premaknemo metek, \v{c}e je bil ta izstreljen.
\begin{lstlisting}[language=c++]
if( bullet.isVisible() ) {
    bullet.setY(bullet.getY() - BULLET_MOVE_SPEED);
    bullet.invalidate();

    if( bullet.getY()+bullet.getHeight() < 0 ) {
        bullet.setVisible(false);
        bullet.invalidate();
    }
}
\end{lstlisting}

\noindent
\v{C}e je igralec pritisnik gumb moramo tudi premakniti ladjo.
\begin{lstlisting}[language=c++]
if( btn_left.getPressedState() && ship.getX() > 0 ) {
    ship.setX(ship.getX() - SHIP_MOVE_SPEED);
}
else if( btn_right.getPressedState() && ship.getX() + ship.getWidth() < SCREEN_WIDTH ) {
    ship.setX(ship.getX() + SHIP_MOVE_SPEED);
}
ship.getParent()->invalidate();
\end{lstlisting}

\noindent
Nasprotnik lahko tudi izstreli metek, vendar je lahko na ekranu
le en nasprotnikov metek hkrati.
\begin{lstlisting}[language=c++]
if( !bullet_enemy.isVisible() ) {
    for(int i = 0; i < NUM_OF_ENEMIES; ++i) {
        if( !enemies[i].isVisible() )
            continue;

        if( pseudo_random(tick) < 10 && tick % 13 == 0 ) {
            bullet_enemy.setVisible(true);
            bullet_enemy.setXY(enemies[i].getX() + enemies[i].getWidth()/2, enemies[i].getY() + enemies[i].getHeight());
            bullet_enemy.invalidate();
            break;
        }
    }
} else {
    bullet_enemy.setY(bullet_enemy.getY() + BULLET_MOVE_SPEED);

    if( bullet_enemy.getY() > SCREEN_HEIGHT ) {
        bullet_enemy.setVisible(false);
    }
    bullet_enemy.invalidate();
}
\end{lstlisting}

\noindent
Premaknemo nasprotnike po vnaprej dolo\v{c}eni poti ter preverimo
\v{c}e je igre konec.
\begin{lstlisting}[language=c++]
static int pixels_moved = 0;

if( ++tick%10 == 0 ) {
    if( enemy_move_direction == MOVE_RIGHT ) {
        if( enemies[NUM_OF_ENEMIES_PER_ROW-1].getX() + enemies[NUM_OF_ENEMIES_PER_ROW].getWidth() < SCREEN_WIDTH ) {
            for(int i = 0; i < NUM_OF_ENEMIES; ++i) {
                enemies[i].setX(enemies[i].getX() + ENEMY_MOVE_SPEED);
                enemies[i].invalidate();
            }
        } else {
            enemy_move_direction = MOVE_DOWN;
        }
    }
    else if( enemy_move_direction == MOVE_LEFT ) {
        if( enemies[0].getX() > 10 ) {
            for(int i = 0; i < NUM_OF_ENEMIES; ++i) {
                enemies[i].setX(enemies[i].getX() - ENEMY_MOVE_SPEED);
                enemies[i].invalidate();
            }
        } else {
            enemy_move_direction = MOVE_DOWN;
        }
    }
    else if( enemy_move_direction == MOVE_DOWN ) {
        if( pixels_moved < enemies[0].getHeight()/2 ) {
            for(int i = 0; i < NUM_OF_ENEMIES; ++i) {
                enemies[i].setY(enemies[i].getY() + ENEMY_MOVE_SPEED);
                enemies[i].invalidate();
                pixels_moved += ENEMY_MOVE_SPEED;
            }
        }
        else {
            pixels_moved = 0;
            if( enemies[NUM_OF_ENEMIES_PER_ROW-1].getX() + enemies[NUM_OF_ENEMIES_PER_ROW].getWidth() < SCREEN_WIDTH ) {
                enemy_move_direction = MOVE_RIGHT;
            }
            else {
                enemy_move_direction = MOVE_LEFT;
            }
        }
    }
}

check_game_over();
\end{lstlisting}

\noindent
\v{C}e je igralec pritisnik gumb za strel ustvarimo metek.
\begin{lstlisting}[language=c++]
void screen_space_invadersView::fire_bullet()
{
	if( !bullet.isVisible() )
	{
		bullet.setXY(ship.getX() + (ship.getWidth()/2 - bullet.getWidth()/2), ship.getY() - bullet.getHeight()/2);
		bullet.setVisible(true);
		bullet.invalidate();
	}
}
\end{lstlisting}

\noindent
Pregledamo \v{c}e je metek zadel katerega od nasprotnikov, \v{c}e je
nasprotnika in metek odstranimo.
\begin{lstlisting}[language=c++]
void screen_space_invadersView::check_bullet_hitbox()
{
	if( bullet.isVisible() ) {
		for(int i = 0; i < NUM_OF_ENEMIES; ++i ) {
			if( !enemies[i].isVisible() )
				continue;

			if( (bullet.getX() < enemies[i].getX()+enemies[i].getWidth()
				&& bullet.getX() > enemies[i].getX()
				&& bullet.getY() > enemies[i].getY()
				&& bullet.getY() < enemies[i].getY() + enemies[i].getHeight())
				|| (bullet.getX() + bullet.getWidth() < enemies[i].getX()+enemies[i].getWidth()
					&& bullet.getX() + bullet.getWidth() > enemies[i].getX()
					&& bullet.getY() > enemies[i].getY()
					&& bullet.getY() < enemies[i].getY() + enemies[i].getHeight())	)
			{
				// The bullet hit an enemy

				// remove the enemy
				enemies[i].setVisible(false);
				enemies[i].invalidate();

				// remove the bullet
				bullet.setVisible(false);
				bullet.setXY(-1, -1);
				bullet.invalidate();

				enemies_killed++;
				break;
			}

		}
	}
}
\end{lstlisting}

\noindent
Funkcija za pregled konca igre.
\begin{lstlisting}[language=c++]
void screen_space_invadersView::check_game_over()
{
	// check that the player hasn't killed all the enemies
	invaders_game_over = true;
	for(int i = 0; i < NUM_OF_ENEMIES; ++i) {
		if( enemies[i].isVisible()) {
			invaders_game_over = false;
			break;
		}
	}

	// check that the enemy hasn't collided with the player
	for(int i = 0; i < NUM_OF_ENEMIES; ++i) {
		if( !enemies[i].isVisible() )
			continue;

		if( (ship.getX() < enemies[i].getX()+enemies[i].getWidth()
			&& ship.getX()+ship.getWidth() > enemies[i].getX()
			&& ship.getY() + ship.getHeight() > enemies[i].getY()
			&& ship.getY() + ship.getHeight()/2 < enemies[i].getY() + enemies[i].getHeight()))
		{
			// Enemy collided with the player
			invaders_game_over = true;
			return;
		}
	}

	// check that the enemy bullet didn't hit the player
	if( (ship.getX() < bullet_enemy.getX()+bullet_enemy.getWidth()
		&& ship.getX()+ship.getWidth() > bullet_enemy.getX()
		&& ship.getY() + ship.getHeight() > bullet_enemy.getY()
		&& ship.getY() + ship.getHeight()/2 < bullet_enemy.getY() + bullet_enemy.getHeight()))
	{
		// Enemy collided with the player
		invaders_game_over = true;
		return;
	}

	// check that the enemy hasn't made it past the player
	for(int i = 0; i < NUM_OF_ENEMIES; ++i) {
		if( !enemies[i].isVisible() )
			continue;

		if( enemies[i].getY() > SCREEN_HEIGHT ) {
			// Enemy has gone past the player
			invaders_game_over = true;
			return;
		}
	}
}
\end{lstlisting}

\noindent
Funkcija za generiranje naklju\v{c}nega \v{s}tevila.
\begin{lstlisting}[language=c++]
int screen_space_invadersView::pseudo_random(int tick)
{
	int bullet_x = 7;
	int bullet_y = 7;

	if( bullet.getX() > 0 )
		bullet_x = bullet.getX();

	if( bullet.getX() > 0 )
		bullet_y = bullet.getY();

	return ((7*tick+13*ship.getX()) + (3*bullet_x+3*bullet_y)) % 100;
}
\end{lstlisting}
\end{document}
